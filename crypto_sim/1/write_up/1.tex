\documentclass{article}

\usepackage{amsmath}
\usepackage{amssymb}

\author{Austin Chase Minor}
\title{Cryptographic Applications of Random Variables}
\date{\today}

\begin{document}
   \maketitle

   \section{Introduction}
   You are a budding cryptoanalyist and have noticed a security
   flaw in your companies login system. You find that matching
   certain keys to values gives you access to peoples account.
   You have no way of determining which key/value pairs match.
   Also, you only have so many retry counts. 
   Your job is to determine the number of key/value matches you
   will get given a certain retry count. This is a mathematical
   hard problem to determine the exact probability since
   as a key/value is matched the number of keys decrease and thus
   the probability increases for a match. So remembering the theory
   of random variables, you code the problem statement and simulate
   to find the average number of matches for a given retry count.
   Below is the mathematical statement of the problem.

   \section{Mathematical Description}
   \begin{flushleft}
      Let $A, B$ be a set of keys and values respectively.\\
      Let $f: A \to B$ be the function relating
      keys to values that you are trying to discover.\\
      Let $\phi: A \times B \to {0,1}$ be a truth
      function representing $true = 1$ if $f(a) = b; a \in A, b \in B$
      and $false = 0$ otherwise.\\
      Let $k \in \mathbb{N}$ represent the
      number of retry counts allowed (the number of tries for each individual
      $b \in B$ to the $\phi$ function.\\
   \end{flushleft}

   \section{Problem Statement}
      Through some thought, it can be shown that the best way to go
      about trying potential values is to try each $b \in B$ with
      every $a \in A k$ times removing $a$ as they are matched. This
      guarantees a minimum of $k$ matches. Furthermore, it maximizes the
      shared information between $b's$ resulting in a higher probability.
      The Matlab code implementing this is below.

      By repeating this trial over k several times, we can generate
      a sample of the actual probability. This allows us to estimate
      the mean and standard deviation along with seeing the general
      shape of the function. The Matlab code for this along with
      the corresponding graphs are below.
   \section{Analysis of Problem}
\end{document}
