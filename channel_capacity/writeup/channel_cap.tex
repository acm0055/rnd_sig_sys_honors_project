\documentclass{article}

\usepackage{amsmath}
\usepackage{minted}

\author{Austin Minor}
\title{Simulation Results of Channel Capacity using Parity Bits for Error Correction}
\date{\today}

\begin{document}
   \maketitle

   \section{Introduction}
      Claude Shannon was a major figure in creating the
      mathematical discipline of information theory. This
      paper analyizes how emperical results line up with
      the statements made by this mathematical model of
      information. More formally, we will be analyizing
      the effect of using parity bits for error correction
      and how well they are at approaching the formal limit
      on the conveyance of data.
   \section{Mathematical Description}
      The official description of converyable information
      is the mutual information between to system of events.
      Let $A$ be the source alphabet. Let $B$ be the reception
      alphabet. Let $Q$ be the transmission matrix (IE the
      probability of receiving $b_j \in B$ given a transmission
      of $a_i \in A$. Let these entries be represented as $q_{i,j}$
      respectively. For our example we will only consider the
      simplier channel known as the binary symettric channel.
      Thus \[ Q =
      \begin{pmatrix}
         q_{0,0} & q_{0,1} \\
         q_{0,1} & q_{1,1}
      \end{pmatrix}
      =
      \begin{pmatrix}
         p & 1-p \\
         1-p & p
      \end{pmatrix}
      \]
      Given such a channel, the probability of bit errors
      can be represented by a Bernoulli Trial.

      The mutual information between two system of events (definition
      given elsewhere) is defined as following:
      \[
         I(A,B) =
         \sum_{i=1}^{m}
         \sum_{j=1}^{n}
         Pr(A_i \cup B_j)
         \log \left(\frac{Pr(A_i \cup B_j)}{Pr(A_i)Pr(B_j)}\right)
      \]
      \[
         ... =
         \sum_{i=1}^{m}
         Pr(A_i)
         \sum_{j=1}^{n}
         q_{i,j}
         \log \left(\frac{q_{i,j}}{\sum_{t=1}^{m} Pr(A_t) q_{t,j}}\right)
      \]
      It has been proved elsewhere that the maximal channel capacity
      (maximum I(A,B) based on $Pr(A_i)$ -- probability that a source letter
      is transmitted) for a BSC is when $Pr(A_1=0)=Pr(A_2=1)=\frac{1}{2}$.
      Furthermore for these values, 
      \[
      I(A,B) = p*log(2*p) + (1-p)*log(2*(1-p))
      \]
      For example, if $p = .99$ then $I(A,B) = .919$.
   \section{Problem Statement}
      To test this model of channel capacity and mutual information,
      a Matlab program was written that would satisfy the important
      requirements listed above ($Pr(A_i) = \frac{1}{2}$, Q as described above).
      The necessary error correction was implemented using parity bits.
      The Matlab code is attached below.

      \begin{listing}[H]
         \inputminted[linenos]{matlab}{../main.m}
         \caption{Main Simulator}
      \end{listing}
      
      \begin{listing}[H]
         \inputminted[linenos]{matlab}{../transmit.m}
         \caption{Transmission Simulator}
      \end{listing}

      \begin{listing}[H]
         \inputminted[linenos]{matlab}{../transmit_msg.m}
         \caption{Per-packet Transmission Simulator}
      \end{listing}

      \begin{listing}[H]
         \inputminted[linenos]{matlab}{../insert_parity_bit.m}
         \caption{Per-packet Parity Bit Inserter}
      \end{listing}

      \begin{listing}[H]
         \inputminted[linenos]{matlab}{../parity.m}
         \caption{Parity Bit Message Inserter}
      \end{listing}

      \begin{listing}[H]
         \inputminted[linenos]{matlab}{../generate_random_msg.m}
         \caption{Generator of Random Messages with Packets}
      \end{listing}
   \section{Results}
      Because the error correction was implemented using parity bits,
      the maximum channel capacity possible is $\frac{4}{5}$. This is since
      one new bit is added to each four bit transmission package thus lengthening
      the transmitted message at a $\frac{4}{5}$ rate.
   \section{Analysis}
\end{document}
