\documentclass{article}

\usepackage{amsmath}
\usepackage{minted}

\author{Austin Chase Minor}
\title{Measuring Room Response and Distance with White Noise}
\date{\today}

\begin{document}
   \section{Introduction}
      In this paper, we will test the use of
      white noise for distance measurement and
      room response. This is a practical application
      of discrete autocorrelation and the FFT
      (Fast Fourier Transform).
   \section{Mathematical Description}
      We know that autocorrelation is a description of
      how correlated a signal is with it self in time.
      This is useful for detecting repetitive responses
      in signals whether by nature of the signal or noise,
      such as echos. White noise, by definition, is a constant
      power signal in the frequency domain. This implies that it
      is only correlated at the origin. We will be using a band-limited
      white noise to approximate this signal.

      We know that the spectral density is the Fourier Transform of the
      the autocorrelation. Since we will be working with a signal in the
      discrete domain, we will use the FFT. This is an fast running
      algorithm to compute the discrete time Fourier Transform. We will
      be able to see what frequencies the room attentuates or accentuates
      by observing the frequency response since white noise has an
      even power across the spectral density.
   \section{Problem Description}
      The basic setup is as pictured above. We will be using a speaker to
      broadcast the white noise, a microphone to pick up the transmitted signal,
      and a wall to measure the distance from. Furthermore, the speaker is
      a reference monitor which implies that the white noise that it has
      a flat frequency response. This is the same with the microphone.
      As in the diagram above, the microphone is facing the back wall. The
      microphone has a hyper-cardioid pickup pattern. This means it will
      pickup noise in the front and back but not sides. Thus we will be
      able to pickup the original white noise and the echo off the back
      wall only. This is of course only an approximatation since the microphone
      merely attentuates signals from other directions. However, the
      attentuatation is large enough to approximate it as described.

      For the measurement of distance to the wall, we set the microphone
      an x-amount of distance from the back wall. For both of the trials this
      was 8' and 9'9" respectively. Then using a band-limited white noise
      signal, generated using the audio program Audacity, we recorded audio
      from the speaker and the relection off the back wall. Then using Matlab
      we calculated the autocorrelation. The code for
      this is below.

      The spectral density measurement involved the same setup as above
      and is included in the code below.

      For both measuring the autocorrelation and spectral density a
      utility program was written to compute the parition of a vector
      into a smaller vector using an average for the partition.

      \begin{listing}
         \inputminted[linenos]{matlab}{../main.m}
         \caption{Main Program}
      \end{listing}
      \begin{listing}
         \inputminted[linenos]{matlab}{../partition.m}
         \caption{Function to Parition Array using Average Scheme}
      \end{listing}
   \section{Analysis}
\end{document}
